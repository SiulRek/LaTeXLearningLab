% Section and frames
\section{APPLIED ML METHODS}
\label{applied_ml_methods}

% Section title slide
\sectiontitleframe{APPLIED ML METHODS}

% Slide 1
\begin{frame}{Regression Methods Overview}
    \frametitle{Regression Methods Overview}
    \begin{itemize}
        \item Linear regression is a fundamental statistical approach for modeling the relationship between a dependent variable and one or more independent variables.
        \item Polynomial regression extends linear regression by considering polynomial features, allowing it to model non-linear relationships.
        \item This approach is effective in revealing hidden patterns and relationships which are not observable in the original linear feature space.
    \end{itemize}
\end{frame}

% Slide 2
\begin{frame}{Polynomial Feature Transformation}
    \frametitle{Polynomial Feature Transformation}
    \begin{itemize}
        \item Polynomial features are created by raising each feature to the power of 2 (degree=2), thereby capturing not just the feature but also its squared effect.
        \item This transformation increases the feature space from the original set to a much larger set, including original features, their squares, and interactions between pairs of features.
        \item In our example the feature space increased from 83 to 3570.
        \item Polynomial regression is suitable for house price prediction as it can account for the complex interplay of features.
    \end{itemize}
\end{frame}

% Slide 3
\begin{frame}[fragile]{Implementing Polynomial Feature Transformation}
    \frametitle{Implementing Polynomial Feature Transformation}
    \begin{lstlisting}[caption={Polynomial feature transformation using \texttt{sklearn.preprocessing}.}, label=lst:polynomial_feature_transformation]
from (*@\module{sklearn.preprocessing}@*) import (*@\class{PolynomialFeatures}@*)
poly_features = (*@\class{PolynomialFeatures}@*)(degree=2)
poly_train = poly_features.fit_transform(X_train_selected)
poly_test = poly_features.transform(X_test_selected)
poly_val = poly_features.transform(X_val_selected)
    \end{lstlisting}
\end{frame}

% Slide 4
\begin{frame}{Implementing Regression Models}
    \frametitle{Implementing Regression Models}
    \begin{itemize}
        \item Implemented Linear, Ridge, and Lasso Regression models using Scikit-learn on polynomial-transformed data.
        \item Explored Ridge and Lasso Regression with varying alpha values (5, 10.0, 50, 100, 500, 1000) to understand the impact of regularization strength.
        \item Evaluated models based on train and validation scores, specifically using R-squared as the performance metric.
        \item Aimed to identify the best performing model with the highest R-squared score.
    \end{itemize}
\end{frame}

%Slide 5
\begin{frame}[fragile]{Linear Regression Model Implementation}
    \frametitle{Linear Regression Model Implementation}
    \begin{lstlisting}[caption={Implementing Linear Regression using sklearn.linear\_model.}, label=lst:linear_regression_implementation]
from (*@\module{sklearn.linear\_model}@*) import (*@\class{LinearRegression}@*)
from (*@\module{sklearn.metrics}@*) import mean_squared_error as mse

# Linear Regression
linear = (*@\class{LinearRegression}@*)()
linear.fit(poly_train, y_train)
train_score = linear.score(poly_train, y_train)
val_score = linear.score(poly_val, y_val)
y_pred = linear.predict(poly_val)
    \end{lstlisting}
\end{frame}


        


% Example frame 1
% \begin{frame}{frame} % set section name
%     \frametitle{frame}
%     \begin{itemize}
%         % Method Description
%         \item 4-5 slides
%         \item Describe and visualise selected ML/DL method (type, logic behind, evaluation performance methodology, framework, reason for selecting this method)
%         % Method Specification
%         \item 1 slide
%         \item Present selected method parameters and hyperparameters…
%     \end{itemize}
% \end{frame}

% Provide the slides here please