% Section and Frames
\section{Framework Requirements and Implementation}
\label{framework_requirements_and_implementation_section}

% Section title frame
\sectiontitleframe{Framework Requirements and Implementation}

\begin{frame}[fragile]{Early Image Preprocessing Efforts}
    \frametitle{Early Image Preprocessing Efforts}
    \begin{lstlisting}[
        caption={Error Prone Implementation of Preprocessing Steps.}, 
        label=lst:histogramequalization
    ]
def global_histogram_equalization(image_tensor, target_tensor):
    cv_img = image_tensor.numpy().astype('uint8')
    channels = (*@\module{cv2}@*).split(cv_img)
    eq_channels = [(*@\module{cv2}@*).equalizeHist(ch) for ch in channels]  
    cv2_eq_image = (*@\module{cv2}@*).merge(eq_channels)
    tf_eq_image = (*@\module{tf}@*).convert_to_tensor(cv2_eq_image, dtype=tf.uint8) 
    return (tf_eq_image, target_tensor)

image_dataset_equalized_glo = image_dataset.map(
    lambda img, tgt: (*@\module{tf}@*).py_function(
        func=global_histogram_equalization, 
        inp=[img, tgt], 
        Tout=((*@\module{tf}@*).uint8, (*@\module{tf}@*).int8)
    ))
    \end{lstlisting}
\end{frame}

\begin{frame}{Framework Requirements}
    \frametitle{Framework Requirements}
    The need for systematic image data preparation motivated the development of the image preprocessing framework. The main requirements for the framework were:
    \vspace{0.5em}
    \begin{itemize}
        \item Sequence flexibility % To ensure smooth transitions between preprocessing steps without misalignments or errors.
        \item Ease of parameter experimentation % To enable dynamic tuning of preprocessing parameters for optimization.
        \item Reproducibility % To allow serialization of steps for consistent application and result comparison.
        \item Modularity % To facilitate the inclusion of preprocessing sequences into larger systems for easy reuse.
    \end{itemize}
\end{frame}

% \begin{frame}{Implementation of Framework Requirements}
%     \frametitle{Implementation of Framework Requirements}
%     % Building on the established requirements, the Image Preprocessing 
%     \textbf{Image Preprocessing Framework key implementations:}
%     \vspace{0.5em}
%     \begin{itemize}
%         \item \textbf{Serialization:} Enables the serialization of complete pipeline configurations, allowing for the easy saving and loading of preprocessing settings.
%         \item \textbf{Parameter Randomization:} Introduces the ability to specify distributions or ranges for parameters, which can then be randomly assigned.
%         \item \textbf{Quality Assurance through Testing:} Usage of a systematic testing approach to ensure each software component functions correctly and integrates flawlessly within the framework.
%     \end{itemize}
% \end{frame}


