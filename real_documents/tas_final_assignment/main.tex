% Preamble


% Run LaTeX compiler multiple times
% First run
\documentclass[12pt]{report}
\usepackage[utf8]{inputenc}
\usepackage[T1]{fontenc}
\usepackage{graphicx}
\usepackage{listings}
\usepackage{amsmath}
\usepackage{float}
\usepackage{hyperref}
\usepackage[nameinlink,noabbrev]{cleveref}
\usepackage{chapterbib}
\usepackage{lmodern}
\usepackage[top=1in]{geometry}
\usepackage{fancyhdr}
\usepackage[numbers]{natbib}
\usepackage{styles/mystyle}

\begin{document}


\begin{titlepage}
    \centering
    \vspace*{1cm}
    
    \Huge
    \textbf{Deep Learning Architectures}
    
    \vspace{0.5cm}
    \LARGE
    A Comprehensive Overview and Guide
    
    \vspace{1.5cm}
    
    \textbf{Your Name}
    
    \vfill
    
    This document presents a detailed exploration \\
    of various deep learning architectures, \\
    their implementations, and applications. \\
    The Document was written using ChatGPT.
    
    \vspace{0.8cm}
    
    \Large
    FH JOANNEUM GmbH\\
    System Test Engineering\\
    Graz\\
    \today
    
    \vspace{2cm}
    
    \includegraphics[width=0.4\textwidth]{figures/logo_fhj_stm.jpg}
    
    \vspace{3cm}
\end{titlepage}


\chapter{Introduction}
This literature review explores Advanced Driver-Assistance Systems (ADAS) and Autonomous Driving (AD), focusing on the vital role of control algorithms in these technologies.

\chapter{Review of Literature}

\section{McKinsey \& Company - ADAS Growth \& Opportunities}
In this report, McKinsey talks about the projected growth in the ADAS hardware sector, especially focusing on processors and optical semiconductors. It mentions how algorithms and software are important in improving sensor performance in various environmental conditions \citep{mckinsey2024}.

\section{ADAS Sensor Technologies}
In this article, there's a deep dive into sensor technologies such as Lidar, Radar, and Ultrasonic sensors in ADAS. It describes their applications, benefits, and limitations, giving a comprehensive perspective on the sensor landscape in ADAS/AD systems \citep{sensorTech2024}.

\section{Tata Elxsi - AD/ADAS Technologies Guide}
In this guide, Tata Elxsi provides a broad overview of AD and ADAS technologies. It discusses sensors like cameras, radar, and LiDAR, along with software algorithms and communication technologies such as V2V and V2I, including a look at emerging trends in automation \citep{tataElxsi2024}.

\section{Speedgoat on ADAS Design \& Testing}
In this resource, Speedgoat offers insights into designing and testing ADAS perception systems. It focuses on the use of tools like Simulink for the development and testing of ADAS algorithms \citep{speedgoat2024}.

\section{MathWorks - ADAS with MATLAB \& Simulink}
In this paper, MathWorks outlines the use of MATLAB in various stages of ADAS development, with a focus on analyzing driving data and developing planning, control, and perception algorithms \citep{mathworks2024}.

\section{NI - Basics of ADAS}
In this article from National Instruments, the basic concepts of ADAS are explained, differentiating between active and passive systems. It also seeks to make clear the differences between ADAS and fully autonomous driving \citep{niADAS2024}.

\section{Recent Trends in ADAS Algorithms}
In this 2024 study, the latest developments in object detection, recognition, and tracking algorithms for ADAS are reviewed. It discusses the evolution of these algorithms and their possible future challenges \citep{adasTrends2024}.

\section{Simulation-Based ADAS Testing}
In this 2017 paper, a simulation-based approach in Software-in-the-Loop environments for testing and validating ADAS systems is advocated for. It points out the complexity of these systems and the need for comprehensive testing methods \citep{simulationTest2024}.

\section{Virtual Testing for ADAS Safety}
In this 2021 paper, the importance of virtual testing for the safety of ADAS algorithms is emphasized. It presents virtual testing as a safer and more cost-effective alternative to real-world testing \citep{virtualTesting2024}.

\section{FEV-Driver for ADAS and AD Development}
In this 2018 paper, FEV-Driver is introduced as a cost-effective platform for developing and testing ADAS and AD functions using an electric go-kart. It describes the platform's application in developing systems like Lane Keep Assist and Automatic Emergency Braking \citep{fevDriver2024}.

\section{Rapid AEB Verification}
In this 2022 paper, a new system for quickly verifying Automatic Emergency Braking algorithms in passenger cars is detailed, focusing on collision prevention and vehicle control \citep{aebVerification2024}.

\chapter{Conclusion}
The literature reviewed highlights the dynamic progression in ADAS/AD technologies, particularly in control algorithms. It underscores ongoing advancements in sensors, software, and testing methodologies in this field.

% Bibliography
\newpage
\bibliography{bibliography/references} % Specify the bibliography file

% End of the document
\end{document}

