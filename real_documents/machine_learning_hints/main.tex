\documentclass{article}
\usepackage[utf8]{inputenc}
\usepackage{geometry}
\usepackage{array}
\usepackage{booktabs}

\geometry{
 a4paper,
 total={170mm,257mm},
 left=20mm,
 top=20mm,
}

\title{Neural Network Configurations for Machine Learning Tasks}
\author{}
\date{}

\begin{document}

\maketitle

\section*{Introduction}
This document outlines typical configurations for neural networks in different machine learning tasks. The configurations vary based on the nature of the task and the complexity of the data. Understanding these configurations is crucial for designing effective neural network models.

\section*{Regression Task}
Regression tasks involve predicting a continuous output variable. The following table provides a general guideline for configuring neural networks for regression tasks. It is important to note that these configurations might need to be adjusted based on the specific characteristics of the dataset and problem.

\begin{tabular}{>{\bfseries}l p{10cm}}
\toprule
Problem Type & Regression (predicting a continuous output) \\
\midrule
Input Layer & Number of neurons corresponds to the number of features in the dataset. \\
\midrule
Hidden Layers & \begin{itemize}
    \item Number of layers: Usually 1-3, but can vary based on complexity.
    \item Number of neurons per layer: Can vary, often starting higher and reducing in deeper layers.
    \item Activation Function: Commonly ReLU (Rectified Linear Unit).
\end{itemize} \\
\midrule
Output Layer & \begin{itemize}
    \item Number of neurons: Usually 1 (for a single continuous output).
    \item Activation Function: None or linear (since it's a regression task).
\end{itemize} \\
\midrule
Loss Function & Mean Squared Error (MSE) or Mean Absolute Error (MAE) are common for regression. \\
\midrule
Optimizer & \begin{itemize}
    \item Often Adam or SGD (Stochastic Gradient Descent).
    \item Learning rate: Needs to be tuned; sometimes learning rate schedulers are used.
\end{itemize} \\
\midrule
Regularization Techniques & \begin{itemize}
    \item Dropout, L1/L2 regularization to prevent overfitting.
    \item Early stopping based on validation loss.
\end{itemize} \\
\midrule
Batch Size and Epochs & \begin{itemize}
    \item Batch size: Varies (common values are 32, 64, 128).
    \item Epochs: Depends on the size of data and the learning rate; use early stopping.
\end{itemize} \\
\midrule
Evaluation Metrics & R\textsuperscript{2} Score, Mean Squared Error, Mean Absolute Error. \\
\midrule
Data Preprocessing & Feature scaling (e.g., normalization or standardization) is important for neural networks. \\
\midrule
Data Splitting & Train, validation, and test split for model training and evaluation. \\
\bottomrule
\end{tabular}

\vspace{1cm}
\section*{Additional Hints}
\begin{itemize}
    \item The architecture of the neural network can significantly impact its performance. Experimentation with different configurations is often necessary.
    \item Regularization techniques are crucial in preventing overfitting, especially in cases where the dataset is not large.
    \item Proper data preprocessing not only speeds up the training process but also improves the model's ability to learn from the data effectively.
    \item Evaluation metrics should be chosen based on the specific goals of the regression task. For instance, MSE is sensitive to outliers, whereas MAE provides a more balanced error metric.
\end{itemize}

\end{document}
