\chapter{ResNeXt}

\section{Description}
ResNeXt is an extension of the ResNet architecture that introduces the concept of aggregated transformations. It maintains the benefits of ResNet's residual connections while increasing the network's cardinality – the size of the set of transformations. This approach allows ResNeXt to achieve higher accuracy without a significant increase in complexity.

\section{Python Implementation}
\begin{lstlisting}[language=Python]
from tensorflow.keras import layers, models

def grouped_convolution_block(input, num_filters, cardinality, strides):
    group_list = []
    for j in range(cardinality):
        x = layers.Conv2D(num_filters // cardinality, kernel_size=3, padding='same', strides=strides, groups=cardinality)(input)
        group_list.append(x)
    x = layers.concatenate(group_list)
    return x

def resnext_block(input, num_filters, cardinality, strides=1):
    shortcut = layers.Conv2D(num_filters, kernel_size=1, strides=strides)(input)
    shortcut = layers.BatchNormalization()(shortcut)

    x = layers.Conv2D(num_filters, kernel_size=1, padding='same')(input)
    x = layers.BatchNormalization()(x)
    x = layers.Activation('relu')(x)

    x = grouped_convolution_block(x, num_filters, cardinality, strides)

    x = layers.Conv2D(num_filters, kernel_size=1, padding='same')(x)
    x = layers.BatchNormalization()(x)
    x = layers.add([shortcut, x])
    x = layers.Activation('relu')(x)
    return x

input = layers.Input(shape=(224, 224, 3))
x = layers.Conv2D(128, kernel_size=7, strides=2, padding='same')(input)
x = layers.MaxPooling2D(pool_size=3, strides=2, padding='same')(x)

for num_filters in [256, 512, 1024, 2048]:
    x = resnext_block(x, num_filters, cardinality=32)

x = layers.GlobalAveragePooling2D()(x)
output = layers.Dense(10, activation='softmax')(x)

model = models.Model(inputs=input, outputs=output)
\end{lstlisting}

\section{Advantages and Disadvantages}
\begin{itemize}
    \item \textbf{Advantages:}
    \begin{enumerate}
        \item Enhanced feature representation due to increased cardinality.
        \item Improved accuracy on image recognition tasks without excessive increase in model complexity.
        \item Residual connections help in training deeper networks effectively.
    \end{enumerate}
    \item \textbf{Disadvantages:}
    \begin{enumerate}
        \item Higher computational requirements compared to simpler architectures.
        \item Increased model complexity can lead to longer training times.
        \item Potential overfitting on smaller datasets without adequate regularization.
    \end{enumerate}
\end{itemize}

\section{Comments}
ResNeXt is a powerful architecture that combines the strengths of increased cardinality with residual learning. Its ability to efficiently process a larger set of transformations makes it a strong candidate for complex image recognition tasks. However, its enhanced capabilities come with the trade-off of increased computational demands, making it suitable for scenarios where resources are not a major constraint.
