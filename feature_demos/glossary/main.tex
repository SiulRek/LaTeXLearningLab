\documentclass{article}
\usepackage[utf8]{inputenc}
\usepackage{imakeidx}

\makeindex[name=glossary, title=Glossary, columns=1]

\begin{document}

\title{An Example Document with a Glossary}
\author{Your Name}
\date{\today}
\maketitle

\section{Introduction}
This document includes a simple glossary. We use \LaTeX\index[glossary]{LaTeX!A high-quality typesetting system; it includes features designed for the production of technical and scientific documentation} for typesetting. 

In this document, we will talk about various topics, including LaTeX, bibliography, mathematics, physics, and algorithms.

\section{Main Content}
\LaTeX\index[glossary]{LaTeX!A high-quality typesetting system} is widely used in academic circles, especially for the creation of documents in mathematics\index[glossary]{Mathematics!The abstract science of number, quantity, and space} and physics\index[glossary]{Physics!The natural science that studies matter, its motion and behavior through space and time, and the related entities of energy and force}. 

An algorithm\index[glossary]{Algorithm!A set of rules to be followed in calculations or other problem-solving operations, especially by a computer} is a fundamental concept in computer science and mathematics.

\section{Conclusion}
A bibliography\index[glossary]{Bibliography!A section of a document that lists references used throughout the work} is essential for academic writing, especially in fields like physics and mathematics.

% Print the glossary
\newpage
\printindex[glossary]

\end{document}
